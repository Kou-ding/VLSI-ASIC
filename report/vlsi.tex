\documentclass[12pt]{report}

% Languages and fonts
\usepackage{polyglossia}
\setdefaultlanguage{greek}
\setotherlanguage{english}
\usepackage{fontspec}
\renewcommand{\thesection}{\arabic{section}} % Begin sections from 1
\setmainfont{Times New Roman} % Set main font to Times New Roman

% Images
\usepackage{graphicx}
\usepackage{float} % placement of figures

% Math
\usepackage{amsmath} % mathematica symbols 
\usepackage{mathrsfs} % curly math letters
\usepackage{amssymb} %more mathematical symbols
\usepackage{gensymb} % degree symbol
\usepackage{mathtools} % For \mathclap

% Miscellaneous
\usepackage{parskip} % Remove paragraph indentation
\usepackage{enumitem} % For custom itemizes
\usepackage{hyperref} % For hyperlinks and references

\begin{document}

\begin{titlepage}
\begin{center}
    \begin{figure}[h]
        \centering
        \includegraphics[width=1\textwidth]{figures/auth_logo.png}
    \end{figure}

    \vspace{1cm}

    \Huge
    \textbf{Σύνθεση και Φυσική Σχεδίαση Επεξεργαστή RISC-V} \\

    \vspace{2cm}
    
    Ψηφιακά Ολοκληρωμένα Κυκλώματα VLSI-ASIC Μεγάλης Κλίμακας \\

    \vspace{2cm}

    \Large
    Παπαδάκης Κωνσταντίνος Φώτιος \\
    ΑΕΜ: 10371

    \vfill

    \large
    Αριστοτέλειο Πανεπιστήμιο Θεσσαλονίκης \\
    Τμήμα Ηλεκτρολόγων Μηχανικών και Μηχανικών Υπολογιστών \\
    \today
\end{center}
\end{titlepage}

\tableofcontents

\newpage

\section{Δημιουργία Script Αυτοματοποίησης}


\section{Άσκηση 1}
Η πρώτη άσκηση αποτελεί την βάση της εργασίας. Καλούμαστε να χρησιμοποιήσουμε τρία εργαλεία της Cadence:
\begin{itemize}
    \item Genus - Σύνθεση
    \item Innovus - Χωροθέτηση, Τοποθέτηση και Δρομολόγηση
    \item Tempus - Ανάλυση Χρονισμού
\end{itemize}
Η υλοποίηση αυτή είναι αρχικά απλοϊκή και έπειτα, μέσω των επόμενων ασκήσεων, εξετάζουμε τροποποιήσεις αυτής ώστε να πληρούνται συγκεκριμένες προδιαγραφές.

Πρώτο βήμα είναι η σύνθεση του ολοκληρωμένου κυκλώματος για τον επεξεργαστή RISC-V.

\end{document}