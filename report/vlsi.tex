\documentclass[12pt]{report}

% Languages and fonts
\usepackage{polyglossia}
\setdefaultlanguage{greek}
\setotherlanguage{english}
\usepackage{fontspec}
\renewcommand{\thesection}{\arabic{section}} % Begin sections from 1
\setmainfont{Times New Roman} % Set main font to Times New Roman

% Images
\usepackage{graphicx}
\usepackage{float} % placement of figures

% Math
\usepackage{amsmath} % mathematica symbols 
\usepackage{mathrsfs} % curly math letters
\usepackage{amssymb} %more mathematical symbols
\usepackage{gensymb} % degree symbol
\usepackage{mathtools} % For \mathclap

% Miscellaneous
\usepackage{parskip} % Remove paragraph indentation
\usepackage{enumitem} % For custom itemizes
\usepackage{hyperref} % For hyperlinks and references

\usepackage{array}
\usepackage{booktabs}
\usepackage{xcolor}
\definecolor{successgreen}{RGB}{34, 139, 34}
\definecolor{warningorange}{RGB}{255, 165, 0}


\begin{document}

\begin{titlepage}
\begin{center}
    \begin{figure}[h]
        \centering
        \includegraphics[width=1\textwidth]{figures/auth_logo.png}
    \end{figure}

    \vspace{1cm}

    \Huge
    \textbf{Σύνθεση και Φυσική Σχεδίαση Επεξεργαστή RISC-V} \\

    \vspace{2cm}
    
    Ψηφιακά Ολοκληρωμένα Κυκλώματα VLSI-ASIC Μεγάλης Κλίμακας \\

    \vspace{2cm}

    \Large
    Παπαδάκης Κωνσταντίνος Φώτιος \\
    ΑΕΜ: 10371

    \vfill

    \large
    Αριστοτέλειο Πανεπιστήμιο Θεσσαλονίκης \\
    Τμήμα Ηλεκτρολόγων Μηχανικών και Μηχανικών Υπολογιστών \\
    \today
\end{center}
\end{titlepage}

\tableofcontents

\newpage

\section{Δημιουργία Script Αυτοματοποίησης}


\section{Άσκηση 1}
Η πρώτη άσκηση αποτελεί την βάση της εργασίας. Καλούμαστε να χρησιμοποιήσουμε τρία εργαλεία της Cadence:
\begin{itemize}
    \item Genus - Σύνθεση
    \item Innovus - Χωροθέτηση, Τοποθέτηση και Δρομολόγηση
    \item Tempus - Ανάλυση Χρονισμού
\end{itemize}
με σκοπό τη σύνθεση του ολοκληρωμένου κυκλώματος για τον επεξεργαστή RISC-V. Η αρχική υλοποίηση είναι απλοϊκή και έπειτα, μέσω των επόμενων ασκήσεων, εξετάζουμε τροποποιήσεις αυτής ώστε να πληρούνται συγκεκριμένες προδιαγραφές.

\subsection{Βήμα 1}
Στο πρώτο βήμα καλούμαστε να 
\begin{itemize}
    \item Ορίσουμε τις βιβλιοθήκες που θα χρησιμοποιήσουμε (timing, lef, qrc)
    \item Διαβάσουμε τον hdl κώδικα του επεξεργαστή RISC-V (picorv32.v) 
    \item Να παρατηρήσουμε και να ερμηνεύσουμε τα console logs
\end{itemize}

\begin{table}[h!]
\centering
\caption{Genus Tool Execution Summary}
\label{tab:genus_summary}
    \begin{tabular}{>{\raggedright\arraybackslash}p{0.25\textwidth} >{\raggedright\arraybackslash}p{0.15\textwidth} >{\raggedright\arraybackslash}p{0.5\textwidth}}
        \toprule
        \textbf{Operation} & \textbf{Status} & \textbf{Details} \\
        \midrule
        Setting library, script and HDL paths & \textcolor{successgreen}{Success} & None \\
        \addlinespace
        Load timing library & \textcolor{warningorange}{Warnings} & Some cells don't have output pins defined \\
        \addlinespace
        Load LEF library & \textcolor{warningorange}{Warnings} & 
        \begin{minipage}[t]{\linewidth}
        \begin{itemize}
            \item Some resistance values are initialized to 0 \\
            \item Some physical cells are not defined in the library \\
            \item According to the LEF library there are total 11 routing layers $[V(5)/H(6)]$ \\
            \item Total of 324 usable logic and 128 usable sequential lib-cells
        \end{itemize}
        \end{minipage} \\
        \addlinespace
        Load QRC library & \textcolor{successgreen}{Success} & 
        According to the QRC library there are total 11 routing layers $[V(5)/H(6)]$ \\
        \addlinespace
        Read HDL picorv32.v & \textcolor{warningorange}{Warnings} & 
        Ignoring unsynthesizable construct $[VLOGPT-37]$ \\
        \bottomrule
    \end{tabular}
\end{table}

\subsection{Βήμα 2}
Στη συνέχεια επεξεργαζόμαστε το κορυφαίο κύκλωμά μας μέσω της εντολής elaborate. Αποσπά το picorv32_wb block από τον κώδικα HDL picorv32.v

\begin{table}[h!]
\centering
\caption{Elaborate}
\label{tab:elaborate_summary}
    \begin{tabular}{>{\raggedright\arraybackslash}p{0.25\textwidth} >{\raggedright\arraybackslash}p{0.15\textwidth} >{\raggedright\arraybackslash}p{0.5\textwidth}}
        \toprule
        \textbf{Operation} & \textbf{Status} & \textbf{Details} \\
        \midrule
        elaborate picorv32\_wb & \textcolor{warningorange}{Warnings} & Removes unused registers. \\
        \addlinespace
        \bottomrule
    \end{tabular}
\end{table}

\end{document}